\chapter{RESULTS}
In this chapter, we present the discussion of our research findings for concurrency and big data datasets. First, we present the results of our sentiment analysis. Next, we analyze the correlations. Finally, we conclude this chapter by answering our research questions \emph{\textbf{R1}} to \emph{\textbf{R4}} with a detailed analysis of the results for both concurrency and big data datasets.

\section{Concurrency and Big Data Sentiment}
In this section, we present our results for concurrency and big data dataset topics sentiment. 

Initially, we calculate the sentiment percentages for each topic within the concurrency dataset, which we presented in Table \ref{tab:CT}. We compute the sentiment percentages for the concurrency topics by evaluating the sentiment for all the questions and answers and take the mean of sentiments measure (positive, negative, neutral) for a topic. 

The results for positive, negative, neutral sentiment percentages regarding the conc- urrency dataset, are shown in Table \ref{tableSentiConcurrency}. For instance, from the table,
if we pick the positive polarity, we find that the \emph{data scraping} topic has the highest rate of positive polarity. In contrast, \emph{process life cycle management} topic has the lowest positive polarity of Stackoverflow questions and answers. The 
\emph{process life cycle management} keeps the lowest sentiment percentage in negative polarity, and \emph{irreproducible behavior} topic claims the highest negative polarity. The same two topics, \emph{process life cycle management} and \emph{irreprod- ucible behavior}, switch their polarities from lowest to highest, highest to lowest respectively in neutral polarities. Overall, concurrency sentiment analysis concludes by evaluating the mean sentiment for all the topics, which are 16.1\%, 23.4\%, and 60.2\% to positive, negative, and neutral sentiment.


Similar to the calculations we did for the concurrency, we compute the big data topics sentiment percentages. The result of our calculations is in Table \ref{tab:BT}. The resulting mean of sentiment measures (positive, negative, neutral) for each topic belonging to the big data dataset is in Table \ref{tab:BT}.

In the big data topic, \emph{basic concepts} topic has the highest rate of positive polarity, and \emph{scala spark} topic has less positively polarized Stackoverflow questions and answers. These two topics switch their polarities from high to low and vice versa for negative polarity. The topic \emph{general programming} has a higher percentage of questions and answers with neutral polarity, whereas \emph{debugging} topic has the lowest. Overall big data sentiment analysis concludes by evaluating the mean sentiment for all the topics, which are 16.8\%, 36.4\%, and 46.6\% to positive, negative, and neutral sentiment, respectively. 

\section{Concurrency and Big Data Sentiment Correlations}

To understand the role of the sentiment towards software developers' productivity, we find the correlations existing between sentiment with popularity and sentiment with difficulty. To do so, we gather the sentiment for each topic within concurrency and big data datasets calculated in the previous section. Then we consider the popularity and difficulty metrics computed by the authors \cite{ahmed2018concurrency, bagherzadeh2019going}, as in Table \ref{CP} and the Table \ref{BP} for concurrency and big data, respectively. Finally, using the tool IBM SPSS, we find the correlations existing in concurrency and big data datasets to answer our research questions \emph{\textbf{R1}} to \emph{\textbf{R4}}.

\begin{table}[tbp]
\caption{Concurrency Topics Sentiment}
\label{tableSentiConcurrency}
\centering
\begin{tabular}{lccc} \hline
\textbf{Topic} & \textbf{Positive} & \textbf{Negative} & \textbf{Neutral} \\ \hline
thread safety & 20.4 & 19.3 & 60.1 \\ 
basic concepts & 19.3 & 20.9 & 59.7 \\ 
task parallelism & 13.5 & 20.0 & 66.3 \\
locking & 9.6 & 18.6 & 71.6 \\ 
thread life cycle management & 10.7 & 22.6 & 66.5 \\ 
thread scheduling & 14.3 & 28.4 & 57.1 \\ 
process life cycle management & 7.6 & 14.0 & 78.3 \\ 
thread pool & 11.6 & 16.6 & 71.7 \\
object-oriented concurrency & 16.4 & 20.1 & 63.4 \\ 
database management systems & 21.7 & 21.4 & 56.8 \\ 
thread sharing & 10.5 & 24.5 & 64.9 \\ 
GUI& 16.7 & 30.7 & 52.4 \\ 
irreproducible behavior & 16.3 & 43.0 & 40.6 \\ 
event-based concurrency & 11.7 & 20.2 & 67.9 \\ 
python multiprocessing & 17.3 & 26.9 & 55.6 \\ 
entity management & 21.5 & 20.1 & 58.2 \\ 
memory consistency & 15.0 & 16.8 & 68.1 \\ 
file management & 12.1 & 18.5 & 69.3 \\ 
producer-consumer concurrency & 14.4 & 23.1 & 62.4 \\ 
unexpected output & 18.3 & 36.9 & 44.7 \\ 
mobile concurrency & 19.4 & 31.1 & 49.4 \\ 
runtime speedup & 16.6 & 20.9 & 62.4 \\ 
web concurrency & 18.2 & 22.6 & 59.1 \\ 
concurrent collections & 20.3 & 16.8 & 62.8 \\ 
client-server concurrency & 16.9 & 30.8 & 52.1 \\ 
data scraping & 26.7 & 27.7 & 45.5 \\ 
parallel computing & 18.8 & 20.8 & 60.3 \\ \hline
\textbf{Average} & \textbf{16.1} & \textbf{23.4} & \textbf{60.2} \\ \hline
\end{tabular}
\end{table}


\begin{table}[tbp]
\caption{Big Data Topics Sentiment}
\label{tab:SentiBigdata}
\centering
\begin{tabular}{lccc} \hline
\textbf{Topic} & \textbf{Positive} & \textbf{Negative} & \textbf{Neutral} \\ \hline
file management & 11.3 & 28.9 & 59.7 \\
file distribution & 15.1 & 38.4 & 46.3 \\ 
hive & 20.6 & 36.7 & 42.6 \\
dependency management & 12.3 & 48.5 & 39.0 \\
dataframe & 20.7 & 27.9 & 51.3 \\
string & 19.6 & 23.7 & 56.5 \\
RDD& 16.2 & 30.6 & 53.1 \\
hbase & 22.4 & 29.2 & 48.2 \\
dataset load\&store & 16.9 & 35.3 & 47.7 \\
data organization & 14.2 & 27.6 & 58.0\\
file format & 15.3 & 29.3 & 55.3 \\
connection management & 15.9 & 43.8 & 40.2 \\
mapreduce model & 21.3 & 26.4 & 52.2 \\
debugging & 15.8 & 54.2 & 29.9 \\
basic concepts & 25.2 & 23.3 & 51.3 \\
pyspark & 13.1 & 51.1 & 35.7 \\
memory management & 15.2 & 38.4 & 46.3 \\
job management & 11.8 & 41.6 & 46.4 \\
general programming & 14.0 & 24.1 & 61.7 \\
PIG & 16.2 & 43.3 & 40.4 \\
date\&time & 20.8 & 29.2 & 49.8 \\
text search & 18.1 & 32.1 & 49.7 \\
scala spark & 6.4 & 54.8 & 38.6 \\
database import\&export & 18.4 & 43.5 & 37.9 \\
performance & 20.9 & 36.6 & 42.3 \\
logging & 17.1 & 40.1 & 42.6 \\
machine learning & 20.1 & 38.6 & 41.2 \\
stream processing & 15.6 & 43.0 & 41.2 \\ \hline
\textbf{Average} & \textbf{16.8} & \textbf{36.4} & \textbf{46.6}\\ \hline
\end{tabular}
\end{table}


\begin{table}[t!hb]
\caption{Correlations in Concurrency}
\label{tab:ConcPopl}
\centering
\begin{tabular}{p{0.7in}p{1.7in}p{0.7in}p{0.7in}p{0.7in}} \hline
\textbf{Metric} & \textbf{Measure} & \textbf{Positive} & \textbf{Negative} & \textbf{Neutral} \\
\hline
popularity & average views & -/.01849 & -/.14996 & +/.07294 \\
\\
& average favorites & -/.19815 & \hl{-/.00154} & +/.00889 \\ 
\\
& average scores & +/.07911 & \hl{-/.00127} & \hl{+/.00320} \\
\\
difficulty & \% w/o acc. answer & +/.08344 & +/.02289 & -/.00758 \\
\\
& hrs. to acc. answer & +/.00731 & +/.08687 & \hl{-/.00565} \\ \hline
\end{tabular}
\end{table}

We discussed the IBM SPSS tool to find the correlations. Using this methodology, the results of our correlations for the concurrency dataset are shown in the Table \ref{tab:ConcPopl}. In our computations, we found a negative correlation exists between the popularity metric's average favorites measure and negative sentiment with a correlation significance at 0.001. This means that, if the questions and answers within a concurrency topic are tilted towards the negative end of the emotions scale, then the users marking these questions and answers as their favorite are fewer. Also, there exists a positive correlation between the popularity metric's average scores measure and neutral sentiment with a significance at 0.003, which means that the neutrally toned questions and answers receive higher scores by the users.

\begin{table}[t!hb]
\caption{Correlations in Big Data}
\label{tab:PoplCorr}
\centering

\begin{tabular}{p{0.7in}p{1.7in}p{0.7in}p{0.7in}p{0.7in}} \hline
\textbf{Metric} & \textbf{Measure} & \textbf{Positive} & \textbf{Negative} & \textbf{Neutral} \\
\hline
popularity & average views & -/.60734 & -/.13308 &.04802 \\
\\
& average favorites & +/.96703 & \hl{-.00292} & \hl{+/.00406} \\ 
\\
& average scores & -/.22997 & -/.07308 & .04983 \\
\\
difficulty & \% w/o acc. answer & -/.13308 & \hl{+/.00006} & \hl{-/.00009} \\
\\
& hrs. to acc. answer & -/.63265 & .00908 & -/.01964 \\ \hline
\end{tabular}
\end{table}

Using the same approach in handling the big data dataset, the results of the correlati- ons we computed were presented in Table \ref{tab:PoplCorr}. Our popularity results indicate that there is a negative and positive correlation between negative and neutral sentiment, respectively, with respect to the average favorites. On the other hand, our difficulty metric results point that there exists a positive and negative correlation between negative and neutral sentiment, respectively, with the percentage of questions with an accepted answer measure. At this point in our analysis, we have the results to answer our research questions.

\section{Discussion}
\label{finalSectionResults}

We answer the research questions \emph{\textbf{R1}} to \emph{\textbf{R4}} introduced at the beginning of this thesis, using the correlations in concurrency and big data datasets, which can be found in the Tables \ref{tab:ConcPopl} and the Table \ref{tab:PoplCorr}. The summary of these correlations are in the Tables \ref{finalCDCorr} and in the Table \ref{finalBDCorr}.

\begin{table}[h!tb]
\caption{Summary of the Concurrency Sentiment Correlations}
\label{finalCDCorr}
\centering
\begin{tabular}{llll} \hline
\textbf{Sentiment} & \textbf{Metric} & \textbf{Direction}&\textbf{Measure}\\ \hline
positive & popularity & & no correlation \\
\\
& difficulty & & no correlation\\
\\
negative & popularity & negative & average favorites\\
\\
& & negative & average scores\\ 
\\
& difficulty & & no correlation \\
\\
neutral & popularity & positive & average scores \\
\\
& difficulty & positive & hrs. to acc. answer \\ \hline
\\
\end{tabular}
\end{table}

\emph{\textbf{R1}}. Is there a relation between sentiment with popularity and sentiment with difficulty in concurrency topics? The summary of the popularity and difficulty metrics and their correlations with sentiment are displayed in Table \ref{finalCDCorr}. For the negative and neutral sentiment, we observe correlations between both popularity and difficulty metrics. However, we did not find any correlations between positive sentiment with both popularity and difficulty metrics.

\emph{\textbf{R2}}. Is there a relation between sentiment with popularity and sentiment with difficulty in big data topics? The summary of the popularity and difficulty metrics and their correlations with sentiment are displayed in Table \ref{finalBDCorr}. For the negative and neutral sentiment, we observe correlations between both popularity and difficulty metrics. No correlation was found between positive sentiment with popularity and difficulty measures. Likewise, during our concurrency dataset analysis, we did not find correlations between positive sentiment with popularity and difficulty.

\begin{table}[b!ht]
\caption{Summary of the Big Data Correlations}
\label{finalBDCorr}
\centering
\begin{tabular}{llll} \hline
\textbf{Sentiment} & \textbf{Metric} & \textbf{Direction}&\textbf{Measure}\\ \hline
positive & popularity & & no correlation \\
\\
& difficulty & & no correlation\\
\\
negative & popularity & negative & average favorites\\
\\
& difficulty & positive & \% w/o acc. answer\\
\\
neutral & popularity & positive & average favorites \\
\\
& difficulty & positive & \% w/o acc. answer \\ \hline
\end{tabular}
\end{table}

\emph{\textbf{R3}}. What sentiment do concurrency and big data developers express towards a common topic? The concurrency and big data sentiment values are displayed in the Table \ref{tab:ConcPopl} and Table \ref{tab:PoplCorr} respectively. The \emph{basic concepts} is a common topic in both concurrency and big data topics. The questions and answers discussed in this topic are generic to the concurrency and big data topics. \emph{Basic concepts} topic has a common property, which is that the questions and answers within these topics are more neutrally toned in both concurrency and big data datasets.

\emph{\textbf{R4}}. What commonality do the correlations hold toward concurrency and big data?
Referring to the Tables \ref{finalCDCorr} and \ref{finalBDCorr}, the commonly maintained property that our computations lead towards a negative sentiment which is inversely correlated with the average favorites measure in popularity metric.