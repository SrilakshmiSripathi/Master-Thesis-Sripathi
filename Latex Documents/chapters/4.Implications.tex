\chapter{IMPLICATIONS}
Studying the sentiment correlations in concurrency and big data datasets, we investi- gate the impact of our findings on different communities. The following are the
implications we deduced using the research questions \emph{\textbf{R1}} to \emph{\textbf{R4}}, regarding specific communities: software practitioners, researchers, and educators. We abbreviated the implications for practitioner's, researcher's and educator's as PI, RI, and EI respectively.

%which we answered in the section \ref{finalSectionResults}.

%Considering the significance of the correlations investigated in this study, we draw the resulting implications and group together according to the needs of researchers and software developers.

%% Significance of Our Findings to Software Development Process
\section {Practitioners}
\emph{\textbf{PI.1}} By referring to the sentiment correlations in the Table \ref{finalCDCorr} and Table \ref{finalBDCorr} consider- ing our research findings for question \emph{\textbf{R4}}. We observed that concurrency and big data developers are sensitive towards negative and neutral sentiment compared to the positive sentiment.

Using the directions observed for negative sentiment with popularity metric, we deduce that concurrency developers' average favorites and average scores measure decreases when they detect more negative sentiment in the Stackoverflow questions and answers. Similarly, big data developers' average favorites measure decreases with the negative sentim- ent in the Stackoverflow questions and answers. Again, using the directions for neutral sentiment with popularity metric, we deduce that concurrency developers' average scores measure rises when a neutral sentiment is detected in the Stackoverflow questions and answers. Likewise, for the big data developers, average favorites measure increases with a neutral sentiment. The difficulty metrics sentiment correlations show that the concurrency developers take longer to receive an accepted answer when they detect neutral sentiment. On the other hand, big data developers will have fewer Stackoverflow questions with accepted answers when they detect both negative and neutral sentiment.

\emph{\textbf{PI.2}} Our findings for research answers to \emph{\textbf{R1}} and \emph{\textbf{R2}} implies that the concurrency and the big data developers receive fewer average favorites measure when Stackoverflow questions and answers have a negative sentiment.

\emph{\textbf{PI.3}} We observed from research answer \emph{\textbf{R3}} that concurrency and big data developers commonly express neutral sentiment towards basic concepts topic.

\section {Researchers}
\emph{\textbf{RI.1}} Our research findings imply correlations between negative and neutral sentiment with popularity and difficulty metrics, refer to Table \ref{finalCDCorr} and Table \ref{finalBDCorr}. However, our research did not find a correlation between positive sentiment with popularity and difficulty with both concurrency and big data topics. Researchers can further explore the sentiment and its effects on various other metrics, not limited to popularity and difficulty. This aggregated research data to understand the human emotions coefficient can further help evolving technology that primarily depends on people.

\section {Educators}
\emph{\textbf{EI.1}} Educators in their practice can refer to the utilization of the Stackoverflow platform while solving popular and difficult topics, especially with concurrency and big data topics. Considering our findings for research question \emph{\textbf{R1}} and the sentiment correlations in the Table \ref{finalCDCorr}, educators, can help concurrency developers recognize the role of negative and neutral sentiment with average favorites, average scores and hours to accepted answer measures. Similarly, our findings for research question \emph{\textbf{R2}} and the sentiment correlations in Table \ref{finalBDCorr}, educators can assist big data developers in understanding the role of negative and neutral sentiment to receive average favorites and percentage of questions with accepted answer measures.