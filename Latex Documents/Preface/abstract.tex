In the current software development environment, a software developer's sentiment towards a challenging task may have an impact on their productivity and the overall quality of their software. Understanding the role of the sentiment can help developers and research communities improve developer's productivity. In this thesis, we conducted a large-scale study by focusing our attention on software developer's sentiment in big data and concurre- ncy software development. Firstly, we use the previous works topic categorization contain- ing 150K questions and answers from 29 big data topics and 250K questions and answers from 27 concurrency topics. Furthermore, we utilize the five measures used in calculating popularity and difficulty metrics by those previous works to understand the role of sentiment. Secondly, we perform sentiment analysis by using the tool Senti4SD to calculate sentiment polarity for each question and answer from those previous works. Finally, using Kendall bivariant correlation, we calculate correlations between popularity with the sentiment and difficulty with the sentiment. Our results indicate that there is a strong inverse correlation between the popularity metric's average favorites measure with the negative sentiment in both concurrency and big data datasets. Considering concurrency findings, we found a negative correlation exists between the popularity metric's average favorites measure with the negative sentiment. Furthermore, there exists a positive correlation between the popularity metric's average scores measure with the neutral sentiment. While our big data findings, imply an inverse and a direct correlation between negative and neutral sentiment with the popularity metric's average favorites measure. Additionally, there exists a direct and an inverse correlation between the negative and the neutral sentiment with the difficulty metric's percentage of questions with accepted answers measure. 


